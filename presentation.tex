%%%%%%%%%%%%%%%%%%%%%%%%%%%%%%%%%%%%%%%%%
% Beamer Presentation
% LaTeX Template
% Version 1.0 (10/11/12)
%
% This template has been downloaded from:
% http://www.LaTeXTemplates.com
%
% License:
% CC BY-NC-SA 3.0 (http://creativecommons.org/licenses/by-nc-sa/3.0/)
%
%%%%%%%%%%%%%%%%%%%%%%%%%%%%%%%%%%%%%%%%%

%----------------------------------------------------------------------------------------
%	PACKAGES AND THEMES
%----------------------------------------------------------------------------------------

\documentclass{beamer}
\usepackage{xcolor}
\usepackage{lmodern}
\usepackage[ruled,vlined]{algorithm2e}

\mode<presentation> {

% The Beamer class comes with a number of default slide themes
% which change the colors and layouts of slides. Below this is a list
% of all the themes, uncomment each in turn to see what they look like.

%\usetheme{default}
%\usetheme{AnnArbor}
%\usetheme{Antibes}
%\usetheme{Bergen}
%\usetheme{Berkeley}
%\usetheme{Berlin}
%\usetheme{Boadilla}
\usetheme{CambridgeUS}
%\usetheme{Copenhagen}
%\usetheme{Darmstadt}
%\usetheme{Dresden}
%\usetheme{Frankfurt}
%\usetheme{Goettingen}
%\usetheme{Hannover}
%\usetheme{Ilmenau}
%\usetheme{JuanLesPins}
%\usetheme{Luebeck}
%\usetheme{Madrid}
%\usetheme{Malmoe}
%\usetheme{Marburg}
%\usetheme{Montpellier}
%\usetheme{PaloAlto}
% \usetheme{Pittsburgh}
%\usetheme{Rochester}
%\usetheme{Singapore}
%\usetheme{Szeged}
%\usetheme{Warsaw}

% As well as themes, the Beamer class has a number of color themes
% for any slide theme. Uncomment each of these in turn to see how it
% changes the colors of your current slide theme.

%\usecolortheme{albatross}
%\usecolortheme{beaver}
%\usecolortheme{beetle}
%\usecolortheme{crane}
\usecolortheme{dolphin}
%\usecolortheme{dove}
%\usecolortheme{fly}
%\usecolortheme{lily}
% \usecolortheme{orchid}
%\usecolortheme{rose}
%\usecolortheme{seagull}
%\usecolortheme{seahorse}
%\usecolortheme{whale}
%\usecolortheme{wolverine}

%\setbeamertemplate{footline} % To remove the footer line in all slides uncomment this line
%\setbeamertemplate{footline}[page number] % To replace the footer line in all slides with a simple slide count uncomment this line

%\setbeamertemplate{navigation symbols}{} % To remove the navigation symbols from the bottom of all slides uncomment this line
\setbeamercovered{transparent}
}

\usepackage{graphicx} % Allows including images
\usepackage{booktabs} % Allows the use of \toprule, \midrule and \bottomrule in tables

%----------------------------------------------------------------------------------------
%	TITLE PAGE
%----------------------------------------------------------------------------------------

\title[WPSS implemented over WebRTC]{The Wormhole Peer Sampling Service implemented over WebRTC} % The short title appears at the bottom of every slide, the full title is only on the title page

\author{Davide Spadini} % Your name
\institute[Unitn] % Your institution as it will appear on the bottom of every slide, may be shorthand to save space
{
University of Trento \\ % Your institution for the title page
\medskip
\textit{davide.spadini@unitn.it} % Your email address
}
\date{\today} % Date, can be changed to a custom date

\begin{document}

\begin{frame}
\titlepage % Print the title page as the first slide
\end{frame}

\begin{frame}
\frametitle{Overview} % Table of contents slide, comment this block out to remove it
\tableofcontents % Throughout your presentation, if you choose to use \section{} and \subsection{} commands, these will automatically be printed on this slide as an overview of your presentation
\end{frame}

%----------------------------------------------------------------------------------------
%	PRESENTATION SLIDES
%----------------------------------------------------------------------------------------

%------------------------------------------------
\section{Wormhole Peer Sampling Service} % Sections can be created in order to organize your presentation into discrete blocks, all sections and subsections are automatically printed in the table of contents as an overview of the talk
%------------------------------------------------

\subsection{Introduction} % A subsection can be created just before a set of slides with a common theme to further break down your presentation into chunks

\begin{frame}
\frametitle{Introduction}
A peer sampling service (PSS) is a service that runs on all the nodes in a distributed system and provides them with a uniform random sample of live nodes from all nodes in the system, where the sample size is typically much smaller than the system size. 
\end{frame}

%------------------------------------------------

\begin{frame}
\frametitle{The idea}
\begin{columns}[t] % The "c" option specifies centered vertical alignment while the "t" option is used for top vertical alignment

\column{.5\textwidth} % Left column and width
\textbf{Past}
\begin{enumerate}
\item Relatively small size
\item (Almost) static network
\item Nodes have {\color{red}{full}} view of the network
\end{enumerate}

\column{.5\textwidth} % Right column and width
\textbf{Now}
\begin{enumerate}
\item Big size
\item Dynamic network
\item Nodes have {\color{red}{partial}} view of the network
\end{enumerate}
\end{columns}\end{frame}

%------------------------------------------------

\begin{frame}
\frametitle{Implementation}
A PSS can be implemented:
\begin{itemize}
	\item as a \textbf{centralized service}: expensive to run reliably
	\item using \textbf{gossip protocols}: the most widely adopted solution
	\item using \textbf{random walks}: suitable for stable networks
\end{itemize}
\end{frame}

\begin{frame}
\frametitle{The problems}
\begin{columns}[t] % The "c" option specifies centered vertical alignment while the "t" option is used for top vertical alignment

\column{.5\textwidth} % Left column and width
\textbf{First problem:}

\begin{itemize}
	\item Network Address Translation (NAT)
	\item Firewall
	\item Antivirus
	\item etc etc..
\end{itemize}

\pause
\textbf{Solution:}

Gossip-based NAT-aware Peer Sampling Service


\column{.5\textwidth} % Right column and width
\textbf{Second problem:}
\begin{itemize}
	\item Require peers to frequently establish network connections and exchange messages with nodes that support direct connectivity
	\item It is a complex and costly procedure!
\end{itemize}
\pause
\textbf{Solution:}

Wormhole Peer Sampling Service!

\end{columns}
\end{frame}

\subsection{WPSS}

\begin{frame}
\begin{itemize}
	\item Published in 2013
	\item It has the same levels of samples' freshness  of the other PSSes, while the connection establishment rate is decreased by one order of magnitude
	\pause
	\item \textbf{Main idea}: to divide the service into two layers
\end{itemize}
\end{frame}

\begin{frame}
\frametitle{Overlays}

\begin{figure}
\includegraphics[keepaspectratio=true, width=\textwidth]{images/overlay}
\end{figure}

\end{frame}

%------------------------------------------------

\subsection{WPSS: The algorithm}
\renewcommand{\thealgocf}{}

\begin{frame}[fragile] % Need to use the fragile option when verbatim is used in the slide
\frametitle{The algorithm}

\begin{algorithm}[H]
  \SetKwProg{Upon}{upon}{ do}{end}
  \SetKwProg{Every}{every}{ do}{end}

  \Upon{wormholeFailure}{
  	$wormhole \leftarrow tracker.getNewWormhole()$\;
    $connect(wormhole)$\;
  }

  \Upon{baseOverlayFailure}{
  	$peer \leftarrow tracker.getNewPeer()$\;
  	$connect(peer)$
  }
 \caption{Wormhole peer sampling}
\end{algorithm}

\end{frame}

\begin{frame}[fragile] % Need to use the fragile option when verbatim is used in the slide
\frametitle{The algorithm}

\begin{algorithm}[H]
  \SetKwProg{Upon}{upon}{ do}{end}
  \SetKwProg{Every}{every}{ do}{end}

  \Every{$\Delta_{wh}$ = wormholeTimeout}{
    $disconnect(wormhole)$\;
  	$wormhole \leftarrow tracker.getNewWormhole()$\;
    $connect(wormhole)$\;
  }

  \Every{$\Delta$ = adTimeout}{
  	$ad \leftarrow createAd()$\;
  	$ad.hops \leftarrow 1$\;
  	$sendAdToWormhole(wormhole, ad)$\;
  }
  \caption{Wormhole peer sampling}
\end{algorithm}

\end{frame}
  
\begin{frame}
\frametitle{The algorithm}
    
\begin{algorithm}[H]
  \SetKwProg{Upon}{upon}{ do}{end}
  \SetKwProg{Every}{every}{ do}{end}

  \Upon{receivedAd(ad)}{
  	\If{ad.hop = getTTL() $\mid\mid$ acceptAd(ad)}{
  		view.addAd(ad)
  	}\Else{
  		$neighbour \leftarrow getMetropolisHastingsNeighbour()$\;
  		$ad.hop \leftarrow ad.hop + 1$\;
  		$sendAd(neighbour, ad)$\;
  	}
  }
  \caption{Wormhole peer sampling}
\end{algorithm}

\end{frame}



%------------------------------------------------
\section{Second Section}
%------------------------------------------------

\begin{frame}
\frametitle{Table}
\begin{table}
\begin{tabular}{l l l}
\toprule
\textbf{Treatments} & \textbf{Response 1} & \textbf{Response 2}\\
\midrule
Treatment 1 & 0.0003262 & 0.562 \\
Treatment 2 & 0.0015681 & 0.910 \\
Treatment 3 & 0.0009271 & 0.296 \\
\bottomrule
\end{tabular}
\caption{Table caption}
\end{table}
\end{frame}

%------------------------------------------------

\begin{frame}
\frametitle{Theorem}
\begin{theorem}[Mass--energy equivalence]
$E = mc^2$
\end{theorem}
\end{frame}

%------------------------------------------------

\begin{frame}[fragile] % Need to use the fragile option when verbatim is used in the slide
\frametitle{Verbatim}
\begin{example}[Theorem Slide Code]
\begin{verbatim}
\begin{frame}
\frametitle{Theorem}
\begin{theorem}[Mass--energy equivalence]
$E = mc^2$
\end{theorem}
\end{frame}\end{verbatim}
\end{example}
\end{frame}

%------------------------------------------------

\begin{frame}
\frametitle{Figure}
Uncomment the code on this slide to include your own image from the same directory as the template .TeX file.
%\begin{figure}
%\includegraphics[width=0.8\linewidth]{test}
%\end{figure}
\end{frame}

%------------------------------------------------

\begin{frame}[fragile] % Need to use the fragile option when verbatim is used in the slide
\frametitle{Citation}
An example of the \verb|\cite| command to cite within the presentation:\\~

This statement requires citation \cite{p1}.
\end{frame}

%------------------------------------------------

\begin{frame}
\frametitle{References}
\footnotesize{
\begin{thebibliography}{99} % Beamer does not support BibTeX so references must be inserted manually as below
\bibitem[Smith, 2012]{p1} John Smith (2012)
\newblock Title of the publication
\newblock \emph{Journal Name} 12(3), 45 -- 678.
\end{thebibliography}
}
\end{frame}

%------------------------------------------------

\begin{frame}
\Huge{\centerline{The End}}
\end{frame}

%----------------------------------------------------------------------------------------

\end{document} 